\documentclass{article}[12pt]
\usepackage{graphicx}
\usepackage{hyperref}
\usepackage[margin=2.5cm]{geometry}
\newcommand{\ignore}[1]{}
\begin{document}
\begin{center}
\textbf{\huge{Getting Started With Picat (Windows Users)}}\\
\end{center}
\rule{450pt}{1pt}

\vspace*{5mm}
\begin{center}
{\Large Bo Yuan Zhou}
\end{center}

\vspace*{5mm}

\section*{\Large{Install Picat}}
\begin{enumerate}
\item Download the zip archive file for Windows from: \url{http://picat-lang.org}.
\item Extract the files of the zip archive to a directory, say \verb+c:\+. The file named \verb+picat.exe+ is the executable of the Picat system.
\item For convenience, add the directory that contains \verb+picat.exe+ to the environment variable \texttt{PATH} so that you can start Picat from any working directory. Please refer to the following Web page for instructions for updating environment variables.
\begin{tabbing}
aa \= aaa \= aaa \= aaa \= aaa \= aaa \= aaa \kill
\> \> \verb+http://www.itechtalk.com/thread3595.html+ 
\end{tabbing}
\end{enumerate}

\ignore{
Once you're finished, the directory that you are in should look like this:
\\
\\
\includegraphics[width=6in]{tutorial_2.jpg}
\\
}

\section*{\Large{Start and Quit Picat}}
\begin{enumerate}
\item Open an OS terminal. This can be done by selecting \verb+Start->Run+ and typing \verb+cmd+ or selecting 
\begin{tabbing}
aa \= aaa \= aaa \= aaa \= aaa \= aaa \= aaa \kill
\> \> \verb+Start->Programs->Accessories->Command Prompt+.  
\end{tabbing}

\item Change the working directory. In the following, the working directory is assumed to be \verb+c:\Picat\work\+. The command DOS \texttt{mkdir} can be used to create directories.

\item Type \verb+picat+ to start the system. If Windows does not recognize the command, then you need to edit the environment variable \texttt{PATH} or type the full path of the executable \verb+picat.exe+. Once Picat is started, the system shows the prompt \verb+Picat>+ and is ready to accept queries.

\item Type \texttt{help} to see the help information.

\item Type \texttt{halt} or \verb+ctrl-d+ (control-d) to exit the system.
\end{enumerate}

\includegraphics[width=6in]{tutorial_1.jpg}

\section*{\Large{Load and Run Programs}} 
\indent For the sake of demonstration we'll create a Picat function to the sum of the
even-valued terms of the Fibonacci sequence whose values do not exceed four million.

\begin{figure}[htb]
\begin{verbatim} 
main =>
    S = 0,
    I = 1,
    F = fib(I),
    while (F <= 4000000)
        if (F mod 2 == 0) then
            S := S+F
        end,
        I := I+1,
        F := fib(I)
    end,
    writef("Sum of the even-valued terms is %w%n",S).
       
main([A1]) => 
    writef("fib(%s)=%w%n",A1,A1.to_integer().fib()).

table
fib(1) = 1.
fib(2) = 2.
fib(N)=fib(N-1)+fib(N-2).
\end{verbatim}
\end{figure}
\vspace*{-5mm}

\begin{enumerate}
\item Create a file named \texttt{fib.pi} in the directory \verb+c:\Picat\work\+.
\item Start Picat.
\item Compile and load the file using \texttt{cl(fib)}.
\item Type \texttt{main} to run the program.
\item You can also call the function {\tt fib} by typing a query such as \verb+X=fib(100)+.
\end{enumerate}

\includegraphics[width=6in]{tutorial_2.jpg}

\clearpage

\section*{\Large{Debug}}
\begin{enumerate}
\item Start Picat.
\item Enable debug mode with \texttt{debug}.
\item Compile and run the file using \texttt{cl(fib)}.
\item Type \texttt{main} to run the program.
\item At the entrance and exit of each call, the debugger displays the call and waits for a command. For the available debugging commands, type the question mark \texttt{?}.
\item Use the command \texttt{spy fib} to set a spy point on the \texttt{fib} function.
\end{enumerate}
Note that only programs compiled in debug mode can be traced or spied on. 

\includegraphics[width=6in]{tutorial_3.jpg}

\includegraphics[width=6in]{tutorial_3_2.jpg}

\section*{\Large{Run Programs Directly}}
\begin{enumerate}
\item Type the DOS command \texttt{picat fib}. The Picat system will execute the \texttt{main/0} predicate defined in \texttt{fib.pi}.

\item Type the DOS command \texttt{picat fib 100}. The Picat system will execute the \texttt{main/1} predicate, which calls the \texttt{fib} function.
\end{enumerate}

\includegraphics[width=6in]{tutorial_4.jpg}

If the command line contains arguments after the file name, then the Picat system calls \texttt{main/1}, passing all the arguments after the file name to the predicate as a list of strings.
\end{document}
