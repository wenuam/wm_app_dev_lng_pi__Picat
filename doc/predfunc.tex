\ignore{
\documentstyle[11pt]{report}
\textwidth 13.7cm
\textheight 21.5cm
\newcommand{\myimp}{\verb+ :- +}
\newcommand{\ignore}[1]{}
\def\definitionname{Definition}

\makeindex
\begin{document}

}
\chapter{\label{chapter:predicates}Predicates and Functions}
In Picat, predicates\index{predicate} and functions\index{function} are defined with rules. Each rule is terminated by a dot (\verb+.+) followed by a white space or the newline character.

Picat has two types of pattern-matching rules: the \emph{non-backtrackable} rule\index{non-backtrackable rule} 
\begin{tabbing}
aa \= aaa \= aaa \= aaa \= aaa \= aaa \= aaa \kill
\> \> $Head, Cond\ $\verb+=>+$\ Body$. 
\end{tabbing}
and the \emph{backtrackable} rule\index{backtrackable rule} 
\begin{tabbing}
aa \= aaa \= aaa \= aaa \= aaa \= aaa \= aaa \kill
\> \> $Head, Cond\ $\verb+?=>+$\ Body$. 
\end{tabbing}
Picat also supports Prolog-style Horn clauses \index{Prolog}\index{Horn clause} and Definite Clause Grammar (DCG) rules \index{Definite Clause Grammar}\index{DCG rule} for predicate definitions. A Horn clause takes the form:
\begin{tabbing}
aa \= aaa \= aaa \= aaa \= aaa \= aaa \= aaa \kill
\> \> $Head\ $\verb+:-+$\ Body$. 
\end{tabbing}
which can be written as $Head$ if $Body$ is \texttt{true}. A DCG rule takes the form:
\begin{tabbing}
aa \= aaa \= aaa \= aaa \= aaa \= aaa \= aaa \kill
\> \> $Head\ $\verb+-->+$\ Body$. 
\end{tabbing}
A predicate definition that consists of Horn clauses can be preceded by an \emph{index declaration}, as described in Section \ref{section:horn}. Picat converts Horn clauses and DCG rules into pattern-matching rules.

\section{Predicates}
A \emph{predicate}\index{predicate} defines a relation, and can have zero, one, or multiple answers. Within a predicate\index{predicate}, the $Head$ is a \emph{pattern} in the form $p(t_1,\ldots,t_n)$, where $p$ is called the predicate\index{predicate} \emph{name}, and $n$ is called the \emph{arity}\index{arity}. When $n=0$, the parentheses can be omitted. The condition $Cond$, which is an optional goal\index{goal}, specifies a condition under which the rule is applicable. $Cond$ cannot succeed more than once. The compiler converts $Cond$ to \texttt{once $Cond$}\index{\texttt{once/1}} if would otherwise be possible for $Cond$ to succeed more than once.

For a call $C$, if $C$ matches the pattern $p(t_1,\ldots,t_n)$ and $Cond$ is true, then the rule is said to be \emph{applicable} to $C$. When applying a rule to call $C$, Picat rewrites $C$ into $Body$. If the used rule is non-backtrackable\index{non-backtrackable rule}, then the rewriting is a commitment, and the program can never backtrack to $C$. However, if the used rule is backtrackable\index{backtrackable rule}, then the program will backtrack to $C$ once $Body$ fails, meaning that $Body$ will be rewritten back to $C$, and the next applicable rule will be tried on $C$. 

A predicate\index{predicate} is said to be \emph{deterministic} if it is defined with non-backtrackable rules\index{non-backtrackable rule} only, \emph{non-deterministic} if at least one of its rules is backtrackable\index{backtrackable rule}, and \emph{globally deterministic} if it is deterministic and all of the predicates\index{predicate} in the bodies of the predicate's\index{predicate} rules are also globally deterministic. A deterministic predicate\index{predicate} that is not globally deterministic can still have more than one answer. 

\subsection*{Example}
\begin{verbatim}
    append(Xs,Ys,Zs) ?=> Xs = [], Ys = Zs.
    append(Xs,Ys,Zs) => Xs = [X|XsR], Zs = [X|ZsR], append(XsR,Ys,ZsR).

    min_max([H],Min,Max) => Min = H, Max = H.
    min_max([H|T],Min,Max) => 
        min_max(T,MinT,MaxT), 
        Min = min(MinT,H),
        Max = max(MaxT,H).
\end{verbatim}
The predicate \texttt{append(Xs,Ys,Zs)}\index{\texttt{append/3}} is true if the concatenation of \texttt{Xs} and \texttt{Ys} is \texttt{Zs}. It defines a relation among the three arguments, and does not assume directionality of any of the arguments. For example, this predicate can be used to concatenate two lists, as in the call 
\begin{verbatim}
    append([a,b],[c,d],L)
\end{verbatim}
this predicate can also be used to split a list nondeterministically into two sublists, as in the call \texttt{append(L1,L2,[a,b,c,d])}\index{\texttt{append/3}}; this predicate can even be called with three free variables, as in the call \texttt{append(L1,L2,L3)}\index{\texttt{append/3}}. 

The predicate \texttt{min\_max(L,Min,Max)} returns two answers through its arguments.  It binds \texttt{Min} to the minimum of list \texttt{L}, and binds \texttt{Max} to the maximum of list \texttt{L}. This predicate does not backtrack. Note that a call fails if the first argument is not a list. Also note that this predicate consumes linear space. A tail-recursive\index{tail recursion} version of this predicate that consumes constant space will be given below.


\section{Functions}
A \emph{function}\index{function} is a special kind of a predicate\index{predicate} that always succeeds with \emph{one} answer. Within a function\index{function}, the $Head$ is an equation $p(t_1,\ldots, t_n)$\verb+=+$X$, where $p$ is called the function\index{function} \emph{name}, and $X$ is an \emph{expression} that gives the return value. Functions\index{function} are defined with non-backtrackable rules\index{non-backtrackable rule} only.  

For a call $C$, if $C$ matches the pattern $p(t_1,\ldots,t_n)$ and $Cond$ is true, then the rule is said to be \emph{applicable} to $C$. When applying a rule to call $C$, Picat rewrites the equation $C$\verb+=+$X'$ into \texttt{($Body$, $X'$=$X$)}, where $X'$ is a newly introduced variable that holds the return value of $C$. 

Picat allows inclusion of \emph{function facts}\index{function fact} in the form {\tt $p$($t_1$,$\ldots$,$t_n$)\verb+=+$Exp$} in function\index{function} definitions. The function fact\index{function fact} {\tt $p$($t_1$,$\ldots$,$t_n$)\verb+=+$Exp$} is shorthand for the rule:
\begin{tabbing}
aa \= aaa \= aaa \= aaa \= aaa \= aaa \= aaa \kill
\> {\tt $p$($t_1$,$\ldots$,$t_n$)\verb+=+$X$ \verb+=>+ $X\verb+=+Exp$.}
\end{tabbing}
where $X$ is a new variable.
 
Although all functions\index{function} can be defined as predicates\index{predicate}, it is preferable to define them as functions\index{function} for two reasons.  Firstly, functions\index{function} often lead to more compact expressions than predicates\index{predicate}, because arguments of function\index{function} calls can be other function\index{function} calls.  Secondly, functions\index{function} are easier to debug than predicates\index{predicate}, because functions\index{function} never fail and never return more than one answer. 

\subsection*{Example}
\begin{verbatim}
    qequation(A,B,C) = (R1,R2), 
        D = B*B-4*A*C, 
        D >= 0 
    => 
        NTwoC = -2*C,
        R1 = NTwoC/(B+sqrt(D)),
        R2 = NTwoC/(B-sqrt(D)).

    rev([]) = [].
    rev([X|Xs]) = rev(Xs)++[X].
\end{verbatim}
The function \texttt{qequation(A,B,C)} returns the pair of roots of \texttt{A*X$^2$+B*X+C=0}. If the discriminant \texttt{B*B-4*A*C} is negative, then an exception will be thrown. 

The function \texttt{rev(L)} returns the reversed list of \texttt{L}. Note that the function \texttt{rev(L)} takes quadratic time and space in the length of \texttt{L}. A tail-recursive\index{tail recursion} version that consumes linear time and space will be given below.

\section{Patterns and Pattern-Matching}
The pattern $p(t_1,\ldots,t_n)$ in the head of a rule takes the same form as a structure. Function\index{function} calls are not allowed in patterns. Also, patterns cannot contain index notations, dot notations, ranges, array comprehensions, or list comprehensions. Pattern matching is used to decide whether a rule is applicable to a call. For a pattern $P$ and a term $T$, term $T$ matches pattern $P$ if $P$ is identical to $T$, or if $P$ can be made identical to $T$ by instantiating $P$'s variables. Note that variables in the term do not get instantiated after the pattern matching. If term $T$ is more general than pattern $P$, then the pattern matching can never succeed.

Unlike calls in many committed-choice languages, calls in Picat are never suspended if they are more general than the head patterns of the rules. A predicate\index{predicate} call fails if it does not match the head pattern of any of the rules in the predicate\index{predicate}. A function\index{function} call throws an exception if it does not match the head pattern of any of the rules in the function\index{function}. For example, for the function call \texttt{rev(L)}, where \texttt{L} is a variable, Picat will throw the following exception:

\begin{tabbing}
aa \= aaa \= aaa \= aaa \= aaa \= aaa \= aaa \kill
\> \> \texttt{unresolved\_function\_call(rev(L))}.
\end{tabbing}

A pattern can contain \emph{as-patterns}\index{as-pattern} in the form \texttt{$V$@$Pattern$}, where $V$ is a new variable in the rule, and $Pattern$ is a non-variable term. The as-pattern\index{as-pattern} \texttt{$V$@$Pattern$} is the same as \texttt{$Pattern$} in pattern matching, but after pattern matching succeeds, $V$ is made to reference the term that matched $Pattern$. As-patterns\index{as-pattern} can avoid re-constructing existing terms.

\subsection*{Example}
\begin{verbatim}
    merge([],Ys) = Ys.
    merge(Xs,[]) = Xs.
    merge([X|Xs],Ys@[Y|_]) = [X|Zs], X<Y => Zs = merge(Xs,Ys). 
    merge(Xs,[Y|Ys]) = [Y|merge(Xs,Ys)].
\end{verbatim}
In the third rule, the as-pattern\index{as-pattern} \texttt{Ys@[Y|\_]} binds two variables: \texttt{Ys} references the second argument, and \texttt{Y} references the car\index{car} of the argument. The rule can be rewritten as follows without using any as-pattern\index{as-pattern}:
\begin{verbatim}
    merge([X|Xs],[Y|Ys]) = [X|Zs], X<Y => Zs = merge(Xs,[Y|Ys]). 
\end{verbatim}
Nevertheless, this version is less efficient, because the cons\index{cons} \texttt{[Y|Ys]} needs to be re-constructed.

\section{Goals}
In a rule, both the condition and the body are \emph{goals}\index{goal}. Queries that the users give to the interpreter are also goals\index{goal}. A goal\index{goal} can take one of the following forms:
\begin{itemize}
\item \texttt{true}\index{\texttt{true}}: This goal\index{goal} is always true.
\item \texttt{fail}\index{\texttt{fail}}: This goal\index{goal} is always false. When \texttt{fail}\index{\texttt{fail}} occurs in a condition, the condition is false, and the rule is never applicable. When \texttt{fail}\index{\texttt{fail}} occurs in a body, it causes execution to backtrack.
\item \texttt{false}\index{\texttt{false}}: This goal is the same as \texttt{fail}.
\item $p(t_1, \ldots, t_n)$: This goal\index{goal} is a predicate call. The arguments $t_1, \ldots, t_n$ are evaluated in the given order, and the resulting call is resolved using the rules in the predicate $p/n$. If the call succeeds, then variables in the call may get instantiated. Many built-in predicates\index{predicate} are written in infix notation. For example, \texttt{X=Y} is the same as \texttt{'='(X,Y)}.
\item \texttt{$P$, $Q$}: This goal\index{goal} is a conjunction of goal\index{goal} $P$ and goal\index{goal} $Q$. It is resolved by first resolving $P$, and then resolving $Q$. The goal\index{goal} is true if both $P$ and $Q$ are true. Note that the order is important: ($P$,$Q$) is in general not the same as ($Q$,$P$).
\item \texttt{$P$ $\&\&$ $Q$}: This is the same as \texttt{($P$,$Q$)}. 
\item \texttt{$P$; $Q$}: This goal\index{goal} is a disjunction of goal\index{goal} $P$ and goal\index{goal} $Q$. It is resolved by first resolving $P$. If $P$ is true, then the disjunction is true. If $P$ is false, then $Q$ is resolved. The disjunction is true if $Q$ is true. The disjunction is false if both $P$ and $Q$ are false. Note that a disjunction can succeed more than once. Note also that the order is important: ($P$; $Q$) is generally not the same as ($Q$; $P$).
\item \texttt{$P$ $|$$|$ $Q$}: This is the same as \texttt{($P$; $Q$)}.
\item \texttt{not $P$}\index{\texttt{not/1}}: This goal\index{goal} is the negation of $P$. It is false if $P$ is true, and true if $P$ is false.  Note a negation goal\index{goal} can never succeed more than once.  Also note that no variables can get instantiated, no matter whether the goal\index{goal} is true or false.
\item \verb-\+- \texttt{$P$}\index{{\verb-\+/1-}}: This is the same as \texttt{not $P$}.
\item \texttt{once $P$}\index{\texttt{once/1}}: This goal\index{goal} is the same as $P$, but can never succeed more than once.
\item \texttt{repeat}\index{\texttt{repeat/0}}: This predicate is defined as follows:
\begin{verbatim}
    repeat ?=> true.
    repeat => repeat.
\end{verbatim}
The \texttt{repeat}\index{\texttt{repeat/0}} predicate is often used to describe failure-driven loops\index{failure-driven loop}. For example, the query 
\begin{verbatim}
      repeat,writeln(a),fail
\end{verbatim} 
repeatedly outputs \texttt{'a'} until \texttt{ctrl-c} is typed.
\item \texttt{if-then}\index{if statement}: An if-then statement takes the form 
\begin{tabbing}
aa \= aaa \= aaa \= aaa \= aaa \= aaa \= aaa \kill
\> \texttt{if $Cond_1$ then} \\
\> \> $Goal_1$ \\
\> \texttt{elseif $Cond_2$ then} \\
\> \> $Goal_2$ \\
\> \> $\vdots$ \\
\> \texttt{elseif $Cond_{n}$ then} \\
\> \> $Goal_{n}$ \\
\> \texttt{else} \\
\> \> $Goal_{else}$ \\
\> \texttt{end}
\end{tabbing}
where the \texttt{elseif} and \texttt{else} clauses are optional. If the \texttt{else} clause is missing, then the else goal\index{goal} is assumed to be \texttt{true}. For the if-then statement, Picat finds the first condition $Cond_i$ that is true. If such a condition is found, then the truth value of the if-then statement is the same as $Goal_i$. If none of the conditions is true, then the truth value of the if-then statement is the same as $Goal_{else}$. Note that no condition can succeed more than once. 
\ignore{
\item \texttt{try-catch}\index{\texttt{try}}: A \texttt{try-catch}\index{\texttt{try}} statement specifies a goal\index{goal} to try, the exceptions that need to be handled when they occur during the execution of the goal\index{goal}, and a clean-up goal\index{goal} that is executed no matter whether the goal\index{goal} succeeds, fails, or is terminated by an exception. The detailed syntax and semantics of the \texttt{try-catch}\index{\texttt{try}} statement will be given in Chapter \ref{chapter:exception} on Exceptions.
}
\item \texttt{throw $Exception$}\index{\texttt{throw/1}}: This predicate throws the term $Exception$. This predicate will be detailed in Chapter \ref{chapter:exception} on Exceptions.

\item \verb+!+\index{cut}: This special predicate, called a \textit{cut}, is provided for controlling backtracking. A cut in the body of a rule has the effect of removing the choice points, or alternative rules, of the goals to the left of the cut. 

\item Loops: Picat has three types of loop statements: foreach, while, and do-while.  A loop statement is true if and only if every iteration of the loop is true. The details of loops are given in Chapter \ref{chapter:loops}.
\end{itemize}

\section{Tail Recursion}
A rule is said to be \emph{tail-recursive}\index{tail recursion} if the last call of the body is the same predicate\index{predicate} as the head. The \emph{last-call optimization}\index{last-call optimization} enables last calls to reuse the stack frame of the head predicate\index{predicate} if the frame is not protected by any choice points. This optimization is especially effective for tail recursion\index{tail recursion}, because it converts recursion into iteration. Tail recursion\index{tail recursion} runs faster and consumes less memory than non-tail recursion.

The trick to convert a predicate\index{predicate} (or a function\index{function}) into tail recursion\index{tail recursion} is to define a helper that uses an \emph{accumulator}\index{accumulator} parameter to accumulate\index{accumulator} the result. When the base case is reached, the accumulator\index{accumulator} is returned. At each iteration, the accumulator\index{accumulator} is updated. Initially, the original predicate\index{predicate} (or function\index{function}) calls the helper with an initial value for the accumulator\index{accumulator} parameter.


\subsection*{Example}
\begin{verbatim}
    min_max([H|T],Min,Max) => 
        min_max_helper([H|T],H,Min,H,Max).

    min_max_helper([],CMin,Min,CMax,Max) => Min = CMin, Max = CMax.
    min_max_helper([H|T],CMin,Min,CMax,Max) => 
        min_max_helper(T,min(CMin,H),Min,max(CMax,H),Max).

    rev([]) = [].
    rev([X|Xs]) = rev_helper(Xs,[X]).

    rev_helper([],R) = R.
    rev_helper([X|Xs],R) = rev_helper(Xs,[X|R]).
\end{verbatim}
In the helper predicate \texttt{min\_max\_helper(L,CMin,Min,CMax,Max)}, \texttt{CMin} and \texttt{CMax} are accumulators\index{accumulator}: \texttt{CMin} is the current minimum value, and \texttt{CMax} is the current maximum value. When \texttt{L} is empty, the accumulators\index{accumulator} are returned by the unification calls \texttt{Min=CMin} and \texttt{Max=CMax}. When \texttt{L} is a cons\index{cons} \texttt{[H|T]}, the accumulators\index{accumulator} are updated: \texttt{CMin} changes to \texttt{min(CMin,H)}\index{\texttt{min/2}}, and \texttt{CMax} changes to \texttt{max(CMax,H)}\index{\texttt{max/2}}. The helper function \texttt{rev\_helper(L,R)} follows the same idea: it uses an accumulator\index{accumulator} list to hold, in reverse order, the elements that have been scanned. When \texttt{L} is empty, the accumulator\index{accumulator} is returned. When \texttt{L} is the cons\index{cons} \texttt{[X|Xs]}, the accumulator\index{accumulator} \texttt{R} changes to \texttt{[X|R]}.
\ignore{
\end{document}
}


